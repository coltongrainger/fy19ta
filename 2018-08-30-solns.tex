\begin{itemize}
\item
  Consider the function \(f(x) := x^2 + 1\). What is the polynomial
  describing \(f(f(x))\)?

  \begin{enumerate}
  \def\labelenumi{(\Alph{enumi})}
  \tightlist
  \item
    \(x^2 + 2\)
  \item
    \(x^4 + x^2 + 1\)
  \item
    \(x^4 + x^2 + 2\)
  \item
    \(x^4 + 2x^2 + 1\)
  \item
    \(x^4 + 2x^2 + 2\)
  \end{enumerate}

  \emph{Answer}: Option (E). Explanation from Vipul Naik:

  \begin{quote}
  We have
  \(f(f(x)) = f(x^2 + 1) = (x^2 + 1)^2 + 1 = x^4 + 2x^2 + 1 + 1\), which
  simplifies to option (E).

  Option (A) is \((x^2 + 1) + 1 = x^2 + 2\). The error here is is not
  squaring \(x^2 + 1\).

  Option (D) is \((x^2 + 1)^2 = x^4 + 2x^2 + 1\). The error here is in
  forgetting to add \(1\).

  Options (B) and (C) are like (D) and (E), with an error in the
  coefficient of \(x^2\).
  \end{quote}
\item
  If \(f(g(x)) = 5\) and \(f(x) = x+3\) for all real \(x\), then
  \(g(x) =\)

  \begin{enumerate}
  \def\labelenumi{(\Alph{enumi})}
  \tightlist
  \item
    \(x-3\)
  \item
    \(3-x\)
  \item
    \(\frac{5}{x+3}\)
  \item
    \(2\)
  \item
    \(8\)
  \end{enumerate}

  \emph{Answer}: (D) \(g(x) = 2\). Explanation:

  \begin{quote}
  With the constant function \(g(x) = 2\), we evaluate
  \(f(g(x)) = f(2) = 5\).
  \end{quote}
\item
  For all positive functions \(f\) and \(g\) of the real variable \(x\),
  let \(\sim\) be a relation defined by
  \[f \sim g \text{ if and only if } \lim_{x\to \infty} \frac{f(x)}{g(x)} = 1.\]
  Which of the following is NOT a consequence of \(f \sim g\)?

  \begin{enumerate}
  \def\labelenumi{(\Alph{enumi})}
  \tightlist
  \item
    \(f^2 \sim g^2\)
  \item
    \(\sqrt{f} \sim \sqrt{g}\)
  \item
    \(e^f \sim e^g\)
  \item
    \(f + g \sim 2g\)
  \item
    \(g \sim f\)
  \end{enumerate}

  \emph{Answer}: (C) \(e^f \sim e^g\). Explanation from Charlie Rambo:

  \begin{quote}
  Let's find a counter example to \(e^f \sim e^g\). Consider
  \(f(x) = x\) and \(g(x) = x+1\). Clearly
  \(\lim_{x\to \infty} \frac{f(x)}{g(x)} = 1.\) But
  \[\lim_{x\to\infty}\frac{e^x}{e^{x+1}} = lim_{x\to\infty}\frac1{e} = \frac1{e} \neq 1.\]
  \end{quote}
\end{itemize}

\hypertarget{references}{%
\subsubsection{References}\label{references}}

\begin{itemize}
\tightlist
\item
  Vipul Naik, \emph{Math 152 Week 1}.
  \url{https://vipulnaik.com/math-152/}.
\item
  GRE Mathematics Test Form GR0568 and Form GR9367.
\end{itemize}
